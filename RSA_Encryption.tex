\documentclass[10pt,fleqn]{article}
\usepackage{amsmath}
\usepackage{fontspec}
\usepackage{geometry}
\geometry{verbose,tmargin=2cm,bmargin=2cm,lmargin=2cm,rmargin=2cm}
\pagestyle{plain}
\usepackage[active]{srcltx}
\usepackage{setspace}
\usepackage{microtype}
\onehalfspacing

\makeatletter

\providecommand\textquotedblplain{%
  \bgroup\addfontfeatures{RawFeature=-tlig}\char34\egroup}

\makeatletter
\@addtoreset{section}{part}
\makeatother

\AtBeginDocument{
  \def\labelitemi{\large\(\star\)}
}

\makeatother


\begin{document}
\title{\textbf{\huge{}RSA ENCRYPTION:}}
\author{AGNI DATTA}

\maketitle
\tableofcontents{}

\pagebreak{}

\part{EULER'S TOTIENT FUNCTION:}

\medskip{}

The totient function $\varphi(n)$, also called Euler's totient function,
is defined as the number of positive integers $\leq n$ that are relatively
prime to (i.e., do not contain any factor in common with) $n$, where
1 is counted as being relatively prime to all numbers. Since a number
less than or equal to and relatively prime to a given number is called
a totative, the totient function $\varphi(n)$ can be simply defined
as the number of totatives of $n$.\\

$\displaystyle\varphi(n)=n\prod_{p|n}\bigg(1-\frac{1}{p}\bigg)$
\\

\subsection{Properties of Euler’s Totient Function:}
\begin{enumerate}
\item For a prime number p, $\varphi(p)= p-1$.
\item For two number $a\:\textrm{and}\:b,\:$ if $gcd(a,b)$ is 1, then
$\varphi(ab)=\varphi(a)\cdot\varphi(b)$
\item For any two prime numbers $p\:$and $q$, $\varphi(pq)=(p-1)\cdot(q-1)$.
This property is used in RSA algorithm. 
\item If $p\:$ is a prime number, then $\varphi(pk)=p^k - p^{k-1}$. This
can be proved using Euler’s product formula.
\item Sum of values of totient functions of all divisors of \textit{n} is equal to \textit{n}.
\end{enumerate}
\pagebreak{}

\part{RSA ENCRYPTION:}

\bigskip{}


\section{ASYMMETRIC ENCRYPTION:}

Asymmetric Encryption uses a mathematically related pair of keys for
encryption and decryption: a public key and a private key. If the
public key is used for encryption, then the related private key is
used for decryption; if the private key is used for encryption, then
the related public key is used for decryption. This is also known
as Public Key Encryption. 

The two participants in the asymmetric encryption workflow are the
sender and the receiver; each has its own pair of public and private
keys. First, the sender obtains the receiver's public key. Next, the
plaintext or ordinary, readable text is encrypted by the sender using
the receiver's public key; this creates ciphertext. The ciphertext
is then sent to the receiver, who decrypts the ciphertext with their
private key and returns it to legible plaintext.

A visualization of how asymmetric cryptography works A visualization
of how public and private keys are used in asymmetric cryptography
Because of the one-way nature of the encryption function, one sender
is unable to read the messages of another sender, even though each
has the public key of the receiver.

\bigskip{}


\section{HISTORY:}

RSA (Rivest--Shamir--Adleman) is a public-key cryptosystem that
is widely used for secure data transmission. It is also one of the
oldest. The acronym RSA comes from the surnames of Ron Rivest, Adi
Shamir and Leonard Adleman, who publicly described the algorithm in
1977. An equivalent system was developed secretly, in 1973 at GCHQ
(the British signals intelligence agency), by the English mathematician
Clifford Cocks. That system was declassified in 1997.

\bigskip{}


\section{DEFINITION:}

In a public-key cryptosystem, the encryption key is public and distinct
from the decryption key, which is kept secret (private). An RSA user
creates and publishes a public key based on two large prime numbers,
along with an auxiliary value. The prime numbers are kept secret.
Messages can be encrypted by anyone, via the public key, but can only
be decoded by someone who knows the prime numbers.

The security of RSA relies on the practical difficulty of factoring
the product of two large prime numbers, the \textquotedbl factoring
problem\textquotedbl . Breaking RSA encryption is known as the RSA
problem. Whether it is as difficult as the factoring problem is an
open question. There are no published methods to defeat the system
if a large enough key is used.

RSA is a relatively slow algorithm. Because of this, it is not commonly
used to directly encrypt user data. More often, RSA is used to transmit
shared keys for symmetric key cryptography, which are then used for
bulk encryption-decryption.

\pagebreak{}

\section{OPERATION:}
\begin{center}
\textbf{The RSA algorithm involves four steps: key generation, key
distribution, encryption, and decryption.}
\par\end{center}

\smallskip{}

A basic principle behind RSA is the observation that it is practical
to find three very large positive integers $e$, $d$, and $n$, such
that with modular exponentiation for all integers \textit{m} (with
0 \leq\textit{ m < n}):

$(m^e)^d \equiv m (\bmod\:m)$

and that knowing $e$ and $n$, or even $m$, it can be extremely
difficult to find $d$. The triple bar ($\equiv$) here denotes modular
congruence.

In addition, for some operations it is convenient that the order of
the two exponentiations can be changed and that this relation also
implies:

$(m^d)^e \equiv m (\bmod\:m)$

\smallskip{}

\begin{center}
RSA involves a \textit{public key} and a \textit{private key}. 
\par\end{center}

The public key can be known by everyone, and it is used for encrypting
messages. The intention is that messages encrypted with the public
key can only be decrypted in a reasonable amount of time by using
the private key. The public key is represented by the integers $n$
and $e$; and, the private key, by the integer $d\:$ (although $n\:$
is also used during the decryption process, so it might be considered
to be a part of the private key, too). $m$ represents the message.

\medskip{}


\section{KEY GENERATION:}

The keys for the RSA algorithm are generated in the following way:
\begin{enumerate}
\item Choose two distinct prime numbers $p$ and $q$
\begin{itemize}
\item For security purposes, the integers $p$ and $q$ should be chosen
at random, and should be similar in magnitude but differ in length
by a few digits to make factoring harder.
\item Prime integers can be efficiently found using a primality test. $p$
and $q$ are kept secret.
\end{itemize}
\item Compute $n = p\cdot q$\\. $n$ is used as the modulus for both the
public and private keys. Its length, usually expressed in bits, is
the key length. $n$ is released as part of the public key.
\end{enumerate}

\end{document}
