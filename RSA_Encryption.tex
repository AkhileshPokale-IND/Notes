\documentclass[10pt,fleqn]{article}
\usepackage{amsmath}
\usepackage{fontspec}
\usepackage{geometry}
\geometry{verbose,tmargin=2cm,bmargin=2cm,lmargin=2cm,rmargin=2cm}
\usepackage{fancyhdr}
\pagestyle{fancy}
\usepackage[active]{srcltx}
\usepackage{setspace}
\usepackage{microtype}
\onehalfspacing

\makeatletter
\setcounter{chapter}{1}

\AtBeginDocument{
  \def\labelitemi{\large\(\star\)}
}

\makeatother


\begin{document}
\title{\textbf{\huge{}RSA ENCRYPTION:}}
\author{AGNI DATTA}

\maketitle
\tableofcontents{}

\pagebreak{}

\part{EULER'S TOTIENT FUNCTION:}

The totient function $\varphi(n)$, also called Euler's totient function,
is defined as the number of positive integers $<=n$ that are relatively
prime to (i.e., do not contain any factor in common with) $n$, where
1 is counted as being relatively prime to all numbers. Since a number
less than or equal to and relatively prime to a given number is called
a totative, the totient function $\varphi(n)$ can be simply defined
as the number of totatives of $n$.\\

$\displaystyle\varphi(n)=n\prod_{p|n}\bigg(1-\frac{1}{p}\bigg)$
\\

\subsection{Some Interesting Properties of Euler’s Totient Function:}
\begin{enumerate}
\item For a prime number p, $\varphi(p)= p-1$.
\item For two number $a\:\textrm{and}\:b,\:$ if $gcd(a,b)$ is 1, then
$\varphi(ab)=\varphi(a)\cdot\varphi(b)$
\item For any two prime numbers $p\:$and $q$, $\varphi(pq)=(p-1)\cdot(q-1)$.
This property is used in RSA algorithm. 
\item If $p\:$ is a prime number, then $\varphi(pk)=p^k - p^{k-1}$. This
can be proved using Euler’s product formula.
\item Sum of values of totient functions of all divisors of \textit{n} is equal to \textit{n}.
\end{enumerate}

\end{document}
