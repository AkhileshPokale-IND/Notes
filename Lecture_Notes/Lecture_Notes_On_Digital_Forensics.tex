\documentclass[10pt,british,english]{article}
\usepackage[T1]{fontenc}
\usepackage[latin9]{luainputenc}
\usepackage{geometry}
\geometry{verbose,tmargin=2cm,bmargin=2cm,lmargin=2cm,rmargin=2cm}
\pagestyle{plain}
\setlength{\parskip}{\medskipamount}
\setlength{\parindent}{0pt}
\usepackage{xcolor}
\usepackage{array}
\usepackage{multirow}
\usepackage{graphicx}
\usepackage{setspace}
\onehalfspacing

\makeatletter

%%%%%%%%%%%%%%%%%%%%%%%%%%%%%% LyX specific LaTeX commands.
%% Because html converters don't know tabularnewline
\providecommand{\tabularnewline}{\\}

%%%%%%%%%%%%%%%%%%%%%%%%%%%%%% User specified LaTeX commands.
\usepackage{multirow}

\AtBeginDocument{
  \def\labelitemi{\large\(\star\)}
}

\makeatother

\usepackage{babel}
\begin{document}
\title{\begin{center}          
   \vspace*{1cm}  
                   
\Huge         
\textbf{LECTURE NOTES ON DIGITAL FORENSICS}
                      
\vspace{0.5cm}          
\LARGE          
Digital Forensics 
                    
\vspace{1.5cm} 
                     
\textbf{AGNI DATTA}
                     
\vfill     
                 
Lecture Notes collected from classes of Dr. Shishir Kumar Shandilya\\          DIGITAL FORENSICS    
                 
\vspace{0.8cm}     
                                   
\Large          
SCHOOL OF COMPUTER SCIENCE\\          
VELLORE INSTITUTE OF TECHNOLOGY\\         
      
  \end{center}}

\maketitle
\pagebreak{}

\tableofcontents{}

\newpage{}

\part{INTRODUCTION TO DIGITAL FORENSICS:}

Digital forensics is the use of scientifically derived and proven
methods towards the preservation, collection, validation, identification,
analysis, interpretation, documentation and presentation of digital
evidence derived from digital sources for the purpose of facilitating
the reconstruction of events found to be criminal, or helping to anticipate
unauthorized actions shown to be disruptive to planned operations.
It involves the application of investigation and analysis techniques
to gather and preserve evidence from a particular computing device
in a way that is suitable for presentation in a court of law. The
goal of computer forensics is to perform a structured investigation
while maintaining a documented chain of evidence to find out exactly
what happened on a computing device and who was responsible for it.

\rule[0.5ex]{1\columnwidth}{1pt}

\section{IMPORTANCE OF COMPUTER FORENSICS:}

The computer has invaded our very existence, become a part of our
lives, and is an integral part of almost every case. Computer Forensics
can help any organization in finding evidence in a variety of cyber
crime cases. Crimes involving a computer can range across the spectrum
of criminal activity, from child pornography to theft of personal
data to destruction of intellectual property. Computer Forensics not
only means recovering deleted files (documents, graphics, etc.), but
also searching the slack and unallocated space on the hard drive-places
where a plethora of evidence regularly resides. It is tracing Windows
artifacts- those tidbits of data left behind by the operating system-for
clues of what the computer has been used for, and, more importantly,
knowing how to find the artifacts, and evaluating the value of information.
Forensic exams allow the processing of hidden files-files that are
not visible or accessible to the user-that contain past usage information.
It is reconstructing and analyzing the date codes for each file-determining
when each file was created, last modified, last accessed and when
deleted. Computer forensics is being able to run a string-search for
e-mail, when no e- mail client is obvious. An analysis will reveal
Internet usage, recover data, and accomplish a full analysis even
after the computer has been formatted. It is using industry-standard
methodology, and with a concise report with clearly demonstrable results,
something that is organized in a manner to make analysis easier.

\rule[0.5ex]{1\columnwidth}{1pt}

\section{LOCARD\textquoteright S EXCHANGE PRINCIPLE:}

Dr. Edmond Locard (13 December 1877 -- 4 April 1966) was a pioneer
in forensic science who became known as the \textquotedbl Sherlock
Holmes of France\textquotedbl . He formulated the basic principle
of forensic science: \textquotedbl\textcolor{red}{Every contact
leaves a trace}\textquotedbl . 

This became known as Locard's exchange principle. \textquotedblleft \textcolor{red}{Anyone,
or anything that enters in the crime scene takes something from the
crime scene with them and leaves something behind, and that both can
be used as forensic evidence}\textquotedblright . Crime reconstruction
involves examining the available physical evidence, those materials
left at or removed from the scene, victim, or offender, for example
fingerprints. These forensically established contacts are then considered
in light of available and reliable witness, the victim, and a suspect's
statements. From this, theories regarding the circumstances of the
crime can be generated and falsified by logically applying the information
of the established facts of the case.

\rule[0.5ex]{1\columnwidth}{1pt}

\section{CARDINAL RULES OF DIGITAL FORENSICS:}

\paragraph{Following are the cardinal rules of the Digital Forensics:}
\begin{itemize}
\item Never mishandle the evidence.
\item Never trust the subject machine or Operating system.
\item Never work on the original evidence.
\end{itemize}
\rule[0.5ex]{1\columnwidth}{1pt}

\section{DIGITAL EVIDENCE:}

Digital evidence is defined as information and data of value to an
investigation that is stored on, received or transmitted by an electronic
device. This evidence can be acquired when electronic devices are
seized and secured for examination. 

Digital evidence: 
\begin{itemize}
\item Is latent (hidden), like fingerprints or DNA evidence,
\item Crosses jurisdictional borders quickly and easily,
\item Can be altered, damaged or destroyed with little effort,
\item Can be time sensitive.
\end{itemize}
\begin{description}
\item [{Admissibility:}] Digital evidence is often ruled inadmissible by
courts because it was obtained without authorization. In most jurisdictions
a warrant is required to seize and investigate digital devices. In
a digital investigation this can present problems where, for example,
evidence of other crimes are identified while investigating another.
Authentication: As with any evidence, the proponent of digital evidence
must lay the proper foundation. Courts largely concerned themselves
with the reliability of such digital evidence. Best evidence rule
is a legal principle that holds an original copy of a document as
superior evidence. The rule specifies that secondary evidence, such
as a copy or facsimile, will be not admissible if an original document
exists and can be obtained. \textquotedbl Secondary evidence\textquotedbl{}
or copies of the content in the original document can be admitted
as evidence. The best evidence rule is only applied in situations
in which a party attempts to substantiate a non-original document
submitted as evidence during a trial. Admissibility of documents before
state court systems may vary. 
\end{description}
Digital Evidence should be the following:
\begin{itemize}
\item Admissible: Conform to legal requirements,
\item Authentic: Relevant to the case,
\item Complete: Not just extracts
\item Reliable: Collected and Handled appropriately,
\item Believable and understandable.
\end{itemize}

\subsection{LEGAL ISSUES:}

MAC details of the files as digital evidence in the seized original
hard disk (hence its image too) must be earlier than the noticing
/ reporting of criminal incident as well as the date \& time of its
seizure. If it is not so, digital evidence will be diagnosed as a
tampered evidence and court can not accept it as an admissible evidence. 

\subsection{TYPES OF DIGITAL EVIDENCE:}

\subsubsection{Volatile (Non-persistent) Memory that loses its contents, as soon
as power is turned off; e.g. Data stored in RAM (semiconductor storage)
(System BIOS: CMOS RAM - battery powered) }

\subsubsection{Non-volatile (Persistent) No change in contents, even if power is
turned off; e.g. Data stored in a tape / hard disk (magnetic storage),
CD / DVD (optical storage), data cards, USB Thumb Drives -- Flash
memory). }

\subsection{ORDER OF VOLATILITY:}

Higher Volatility to Lower Volatility List:
\begin{enumerate}
\item Registers \& Cache
\item Routing Tables
\item ARP Cache
\item Process Table
\item Kernel Statistics \& Modules 
\item Main Memory (RAM)
\item Temporary System files
\item Secondary Memory
\item Router Configuration
\item Network Topology 
\end{enumerate}

\subsection{IMPORTANCE OF ORDER OF VOLATILITY:}

Current running state and system configuration details.
\begin{itemize}
\item Activities performed/ In progress
\item Root cause of the incident
\item Timeline of the incident
\item Time, date, user responsible for the incident
\item Network connection details
\end{itemize}
Once system is shutdown / rebooted volatile data is lost for ever. 

\rule[0.5ex]{1\columnwidth}{1pt}

\section{PROCEDURE OF COLLECTING DIGITAL EVIDENCE:}

\subsection{ORDER OF HANDLING DATA IN CRIME SCENE:}
\begin{itemize}
\item Document the Crime Scene - OS (Version),
\item BIOS date \& time (and difference, if any),
\item H/W and S/W Configuration,
\item IP / MAC address,
\item Computer System : Shutdown / Power Off ?
\item Identify Evidence \& Authenticate through a Hashing Algorithm (MD5),
\item Always make the bit-stream copy (forensic image) of the seized storage
media,
\item Label all the connecting cables and photograph them,
\item Document the chain of custody,
\item Preserve the evidence before packing for transportation,
\item Securely pack \& transport the evidence to lab,
\item Store the seized evidence in a protected storage (air bubbled PVC,
anti-static bag),
\item Transfer the Computer System to a secure location.
\end{itemize}

\subsubsection{DOCUMENT EVERYTHING:}

The Crime Scene Computer forensics is a meticulous practice. When
a crime involving electronics is suspected, a computer forensics investigator
takes each of the following steps to reach --- hopefully --- a successful
conclusion: Secure the area, which may be a crime scene. Document
the chain of custody of every item that was seized. If someone is
already working on the system then ask/ insist him/her to leave the
terminal. Take photographs, note down the important things, if any.
Bag, tag, and safely transport the equipment and e-evidence. Acquire
the e-evidence from the equipment by using forensically sound methods
and tools to create a forensic image of the e-evidence. Keep the original
material in a safe, secured location. Design your review strategy
of the e-evidence, including lists of keywords and search terms. Examine
and analyze forensic images of the e-evidence (never the original!)
according to your strategy. Interpret and draw inferences based on
facts gathered from the e-evidence. Describe your analysis and findings
in an easy-to-understand and clearly written report. Give testimony
under oath in a deposition or courtroom. 

\subsubsection{SECURING THE CRIME SCENE:}

Secure the area containing the equipment or the crime scene. Secure
the entrances and exits to the digital scene. Move people away from
computer and power supply as it may lead to contamination if anyone
touches anything. Preventing changes in potential digital evidence,
including network isolation, collecting volatile data, and copying
entire digital environment is the goal of this phase. If there are
any ongoing processes, they have to be captured so that to not cause
loss of potential evidence.

\subsubsection{IDENTIFYING THE EVIDENCE SOURCES:}

Generating a plan of action to conduct an effective digital investigation,
and obtaining supporting resources and materials is a part of this
phase. Recognizing an incident from indicators and determining its
type, which entails the preparation of tools, techniques, search warrants,
and monitoring authorizations and management support. Finding potential
sources of digital evidence (e.g., at a crime scene, within an organization,
or on the Internet) is done.

\paragraph{What should be seized for the retrieval of evidence?}

\subparagraph{Examples:}
\begin{itemize}
\item Main unit, usually the box to which the monitor and keyboard are attached.
\item Monitor, keyboard and mouse (only necessary in certain cases.)
\item Power supply units.
\item Hard disks not inside the computer.
\item Dongles
\item Modems (some contain phone numbers).
\item External drives and other external devices.
\item Routers. 
\item Digital cameras.
\item Back up tapes.
\item CDs/ DVDs.
\item Memory sticks, memory cards and all USB connected devices.
\end{itemize}

\subsubsection{DOCUMENT THE SCENE:}

Photographs or videos of digital evidence are taken individually as
well as crime scene and individuated descriptions of digital evidences
are to be made. Each piece of digital evidence that is found during
the analysis of the image must be clearly documented. Proper thorough
chain of custody has to be maintained. Chain of custody is a form
which documents the movement of evidence from its source to when it
is presented in court.

It is essential that any items of evidence can be traced from the
crime scene to the court room, and everywhere in between, known as
\textcolor{red}{CHAIN OF CUSTODY}. 
\begin{itemize}
\item Positive control is the phrase most often used to describe the standard
of care taken in the handling of potential evidentiary material (e.g.,
suspect computer systems, hard drives, and any backup copies). 
\item Who handled the evidence? 
\item What procedures were performed on the evidence? 
\item When was the evidence collected and/or transferred to another party? 
\item Where was the evidence collected and stored? 
\item How was the evidence collected and stored? 
\item For what purpose was the evidence collected?
\end{itemize}

\subsubsection{ACQUIRE THE EVIDENCE:}

Analysis should always take place on a forensically sound copy, or
image, of the seized data, rather than the original data itself. For
this purpose a bit by bit copy of the evidence is to be made. The
storage device is first connected to a \textquotedblleft write blocking\textquotedblright{}
device, which prevents any binary code from being altered or modified
during the process. Then a mirror image or \textquotedblleft clone\textquotedblright{}
of the drive is created on a separate storage device to be examined
later. During the acquisition process, such software creates a unique
numerical code, called a verification \textquotedblleft hash\textquotedblright{}
of the media, which allows an analyst to later confirm that the image
and its contents are accurate and unaltered. Data acquisition can
be carried out either online (live) or offline (dead). A dead acquisition
is carried out without the support of the suspect\textquoteright s
operating system while a live acquisition is carried out with the
support of suspect operating system.

\subsubsection{PHOTOGRAPHY AND VIDEOGRAPHY:}

Forensic photography, also referred to crime scene photography, is
an activity that records the initial appearance of the crime scene
and physical evidence, in order to provide a permanent record for
the courts and for the record. Proper documentation is to be done
and the evidences , the crime scene is to be photographed with a view
to properly locate the things in and around the vicinity of the crime
scene.

The following details/ steps need to be photographed sequentially
with proper scale and labeling: When entering the scene of crime,
photograph the scene to record the details of the crime scene. Photograph
the live status of the system found at the scene, this includes current
applications opened, cables/ USB attached, any running processes etc.
After collection, the evidence is photographed with and without a
scale and before and after packaging.

\rule[0.5ex]{1\columnwidth}{1pt}

\section{COMMUNITIES\textquoteright{} WORKS IN DIGITAL FORENSICS:}

\paragraph{Communities working in Digital Forensics:}
\begin{itemize}
\item Law Enforcement (i.e. Police)
\item Military /Intelligence
\item Business \& Industry
\item Academia/ Researchers 
\end{itemize}
\rule[0.5ex]{1\columnwidth}{1pt}

\section{USES OF DIGITAL FORENSICS:}

\paragraph*{Digital forensics can be used in a variety of settings, including}
\begin{itemize}
\item Criminal Investigations
\item Civil Litigation
\item Intelligence
\item Administrative Matters
\end{itemize}

\subsection{CRIMINAL INVESTIGATION:}

\paragraph{In today\textquoteright s digital world, electronic evidence can
be found in almost any criminal investigation conducted. }
\begin{itemize}
\item Homicide, physical assault, robbery, and burglary are just a few of
the many examples of \textquotedblleft analog\textquotedblright{}
crimes that can leave digital evidence.
\item One of the major struggles in law enforcement is to change the paradigm
of the police and get them to think of and seek out digital evidence. 
\item Everyday digital devices such as cell phones and gaming consoles can
hold a treasure trove of evidence. 
\item Unfortunately, none of that evidence will ever see a courtroom if
it\textquoteright s not first recognized and collected.
\item As time moves on and our law enforcement agencies are replenished
with \textquotedblleft younger blood,\textquotedblright{} this will
become less and less of a problem.
\end{itemize}

\subsection{CIVIL LITIGATION:}

Civil litigation is a legal process in which criminal charges and
penalties are not at issue. When two or more parties become embroiled
in such a non-criminal legal dispute, the case is presented at a trial
where plaintiffs seek compensation or other damages from defendants.

The use of digital forensics in civil cases is big business. In 2011,
the estimated total worth of the electronic discovery market is somewhere
north of \$780 million (Global EDD Group). As part of a process known
as Electronic Discovery (eDiscovery), digital forensics has become
a major component of much high dollar litigation. eDiscovery \textquotedblleft refers
to any process in which electronic data is sought, located, secured,
and searched with the intent of using it as evidence in a civil or
criminal legal case\textquotedblright{} In a civil case, both parties
are generally entitled to examine the evidence that will be used against
them prior to trial. This legal process is known as \textquotedblleft discovery.\textquotedblright{}
Previously, discovery was largely a paper-based exercise, with each
party exchanging reports, letters, and memos; however, the introduction
of digital forensics and eDiscovery has greatly changed this practice.
The proliferation of the computer has rendered that practice nearly
extinct. Today, parties no longer talk about filing cabinets, ledgers,
and memos; they talk about hard drives, spreadsheets, and file types.
Some paper-based materials may come into play, but it\textquoteright s
more the exception than the rule. Seeing the evidentiary landscape
rapidly changing, the courts have begun to modify the rules of evidence.
The rules of evidence, be they state or federal rules, govern how
digital evidence can be admitted during civil litigation. The Federal
Rules of Civil Procedure were changed in December 2006 to specifically
address how electronically stored information is to be handled in
these cases. Digital evidence can quickly become the focal point of
a case, no matter what kind of legal proceeding it\textquoteright s
used in. The legal system and all its players are struggling to deal
with this new reality.

\subsection{INTELLIGENCE:}
\begin{itemize}
\item Terrorists and foreign governments, the purview of our intelligence
agencies, have also joined the digital age. 
\item Terrorists have been using information technology to communicate,
recruit, and plan attacks. 
\item The armed forces are exploiting intelligence collected from digital
devices brought straight from the battlefield. 
\item This process is known as DOMEX (Document and Media Exploitation). 
\item DOMEX is paying large dividends, providing actionable intelligence
to support the soldiers on the ground army.
\end{itemize}

\subsection{ADMINISTRATIVE MATTERS:}
\begin{itemize}
\item Digital evidence can also be valuable for incidents other than litigation
and matters of national security. 
\item Violations of policy and procedure often involve some type of electronically
stored information, for example, an employee operating a personal
side business, using company computers while on company time. 
\item That may not constitute a violation of the law, but it may warrant
an investigation by the company.
\end{itemize}
\rule[0.5ex]{1\columnwidth}{1pt}

\section{ROLE OF THE FORENSIC EXAMINER IN THE JUDICIAL SYSTEM:}
\begin{itemize}
\item The digital forensics practitioner most often plays the role of an
expert witness.
\item What makes them different than non expert witnesses? Other witnesses
can only testify to what they did or saw. They are generally limited
to those areas and not permitted to render an opinion. 
\item Experts, by contrast, can and often do give their opinion. What makes
someone an \textquotedblleft expert?\textquotedblright{} In the legal
sense, it\textquoteright s someone who can assist the judge or jury
to understand and interpret evidence they may be unfamiliar with. 
\item To be considered an expert in a court of law, one does not have to
possess an advanced academic degree. 
\item An expert simply must know more about a particular subject than the
average lay person.
\item Under the legal definition, a doctor, scientist, baker, or garbage
collector could be qualified as an expert witness in a court of law.
Individuals are qualified as experts by the court based on their training,
experience, education, and so on.
\end{itemize}

\subsection{What separates a qualified expert from a truly effective one? }
\begin{itemize}
\item It is their ability to communicate with the judge and jury. 
\item They must be effective teachers. 
\item The vast majority of society lacks technical understanding to fully
grasp this kind of testimony without at least some explanation. 
\item Digital forensic examiners must carry out their duties without bias.
Lastly, a digital forensics examiner must go where the evidence takes
them without any preconceived notions.
\end{itemize}
\rule[0.5ex]{1\columnwidth}{1pt}

\section{CHALLENGES IN DIGITAL FORENSICS:}

\subsection{TYPES OF EVIDENCES:}

\subsubsection{Real Evidence: Real evidence is any evidence that speaks for itself
without relying on anything else. In electronic terms, this can be
a log produced by an audit function---provided that the log can be
shown to be free from contamination.}

\subsubsection{Testimonial Evidence: Testimonial evidence is any evidence supplied
by a witness. This type of evidence is subject to the perceived reliability
of the witness, but as long as the witness can be considered reliable,
testimonial evidence can be almost as powerful as real evidence. Word
processor documents written by a witness may be considered testimonial---as
long as the author is willing to state that they wrote it.}

\subsubsection{Hearsay: Hearsay is any evidence presented by a person who was not
a direct witness. Word processor documents written by someone without
direct knowledge of the incident is hearsay. Hearsay is generally
inadmissible in court and should be avoided.}

\pagebreak{}

\subsection{EVIDENCE HANDLING:}

Evidence handling has four primary areas in any incident response
activity. 

These areas are:

\subsubsection{Collection, which has to do with searching for evidence, recognition
and collection of evidence, and documenting the items of evidence.
Always ensure the collection includes all of the available data and
resources, such as the whole disk drive, not just the used portions.
Always document the place, time, and circumstances of each data item
collected for evidence.}

\subsubsection{Hardware evidence examination, which has to do with origins, significance,
and visibility of evidence, often can reveal hidden or obscured information
and documentation about the evidence. Dimensions, styles, sizes, and
manufacturer of hard drives, other devices, or network items are all
important evidence items.}

\subsubsection{Software and network evidence analysis, which is where the logs/records/software
evidence is actually examined for the incident providing the significance
criteria for inclusion and the probative value of the evidence. Always
conduct this software and network analysis and interpretation separate
from the hardware evidence examination.}

\subsubsection{Evidence reporting, it must be written documentation with the processes
and procedures outlined and explained in detail in the reports. Pertinent
facts and data recovered are the primary keys in the reports. Understand
the documentation and reports will always be reviewed, critiqued,
and maybe even cross-examined.}

\pagebreak{}

\subsection{TECHNICAL CHALLENGES IN EVIDENCE HANDLING:}

As technology develops crimes and criminals are also developed with
it. Digital forensic experts use forensic tools for collecting shreds
of evidence against criminals and criminals use such tools for hiding,
altering or removing the traces of their crime, in digital forensic
this process is called Anti- forensics technique which is considered
as a major challenge in digital forensics world. 

\paragraph*{Anti-forensics techniques are categorized into the following types:}

\subsubsection{ENCRYPTION: It is legitimately used for ensuring the privacy of information
by keeping it hidden from an unauthorized user/person. This helps
protect the confidentiality of digital data either stored on computer
systems or transmitted through a network like the internet. Unfortunately,
it can also be used by criminals to hide their crimes.}

\subsubsection{DATA HIDING IN STORAGE SPACE: Criminals usually hide chunks of data
inside the storage medium in invisible form by using system commands,
and programs.}

\subsubsection{COVERT CHANNEL: A covert channel is a communication protocol which
allows an attacker to bypass intrusion detection technique and hide
data over the network. The attacker used it for hiding the connection
between him and the compromised system.}

\subsubsection{CLOUD OPERATION: This allows the hackers to spoof their IP \foreignlanguage{british}{addresses}
and also makes forensics expert difficult tracking the actual presence
of machine through which the malicious attack has been occurring.}

\subsubsection{ARCHIVAL TIME: This small gap in time is crucial as it allows the
offender to clean their digital trace and also destroy important evidence
pertaining to the case. This time can also be the difference between
apprehending a suspect and for him to be going free.}

\subsubsection{SKILL GAP: This mainly occurs when a digital forensic expert is lacking
experience and knowledge when compared to their illegal counterparts.
This is a major issue between the current landscape, as Black Hat
and White Hat hackers have quite a known skill difference.}

\subsubsection{STEGANOGRAPHY: Steganography is the technique of hiding secret data
within an ordinary, non-secret, file or message in order to avoid
detection; the secret data is then extracted at its destination. The
use of Steganography can be combined with encryption as an extra step
for hiding or protecting data.}

\pagebreak{}

\subsection{LEGAL CHALLENGES IN DIGITAL FORENSICS:}

The presentation of digital evidence is more difficult than its collection
because there are many instances where the legal framework acquires
a soft approach and does not recognize every aspect of cyber forensics.
Besides, most of the time electronic evidence is challenged in the
court due to its integrity. In the absence of proper guidelines and
the nonexistence of proper explanation of the collection, and acquisition
of electronic evidence gets dismissed in itself.

\subsubsection{ABSENCE OF GUIDELINES AND STANDARDS: In India, there are no proper
guidelines for the collection and acquisition of digital evidence.
The investigating agencies and forensic laboratories are working on
the guidelines of their own. Due to this, the potential of digital
evidence has been destroyed. }

\subsubsection{LIMITATION OF THE INDIAN EVIDENCE ACT, 1872: The Indian Evidence
Act, 1872 have limited approach, it is not able to evolve with the
time and address the E-evidence are more susceptible to tampering,
alteration, transposition, etc. the Act is silent on the method of
collection of e-evidence it only focuses on the presentation of electronic
evidence in the court by accompanying a certificate as per subsection
4 of Sec. 65B. This means no matter what procedure is followed it
must be proved with the help of a certificate.}

\subsubsection{PRIVACY ISSUES: A major concern for people those who are affected
by an incident. They do not want to raise a complaint due to the issue
in privacy. This leads to a legal barrier between the process of evidence
handling and evidence presentation in court.}

\subsubsection{ADMISSIBILITY IN COURTS: Due to lack of proper polices, the data
collected can be proven inadmissible if any fault is found in the
process of data accumulation. }

\subsubsection{PRESERVATION OF ELECTRONIC EVIDENCE: A major concern in the techniques
of preservation used to store the digital evidence. Any changes and
manipulations in the evidence will render the digital evidence inadmissible
in court.}

\subsubsection{POWER FOR GATHERING DIGITAL EVIDENCE: The digital forensic expert
needs enough computing, legal and man power for the proper collection
and preservation of data.}

\subsubsection{ANALYZING A RUNNING COMPUTER: Analysis of a running computer is not
easy and data collection can also cause changes in the current state
of the running machine. The order of volatility needs to be followed
otherwise there will a huge change the data collected. This will be
inadmissible in court.}

\pagebreak{}

\subsection{RESOURCE CHALLENGES IN DIGITAL FORENSICS:}

As the rate of crime increases the number of data increases and the
burden to analyze such huge data is also increases on a digital forensic
expert because digital evidence is more sensitive as compared to physical
evidence it can easily disappear. For making the investigation process
fast and useful forensic experts use various tools to check the authenticity
of the data but dealing with these tools is also a challenge in itself. 

\subsubsection{CHANGE IN TECHNOLOGY: Due to rapid change in technology like operating
systems, application software and hardware, reading of digital evidence
becoming more difficult because new version software\textquoteright s
are not supported to an older version and the software developing
companies did provide any backward compatible\textquoteright s which
also affects legally.}

\subsubsection{VOLUME AND REPLICATION: The confidentiality, availability, and integrity
of electronic documents are easily get manipulated. The combination
of wide-area networks and the internet form a big network that allows
flowing data beyond the physical boundaries. Such easiness of communication
and availability of electronic document increases the volume of data
which also create difficulty in the identification of original and
relevant data.}

\rule[0.5ex]{1\columnwidth}{1pt}

\section{A FLOW OF EVIDENCE DIAGRAM:}

\includegraphics[scale=0.5]{\string"Pictures/LNDF (1)\string".png}

\rule[0.5ex]{1\columnwidth}{1pt}

\part{INCIDENT RESPONSE: }

Incident Response is an organized approach to addressing and managing
the aftermath of a security breach or cyber-attack.

\section{INCIDENT RESPONSE:}

Incident response is a key component of an enterprise business continuity
and resilience program. 

The increasing number and diversity of information security threats
can disrupt enterprise business activities and damage enterprise information
assets. 

A sound risk management program can help reduce the number of incidents,
but there are some incidents that can neither be anticipated nor avoided. 

Therefore, the enterprise needs to have an incident response capability
to detect incidents quickly, contain them, mitigate impact, and restore
and reconstitute services in a trusted manner.

\rule[0.5ex]{1\columnwidth}{1pt}

\section{INCIDENT: }
\begin{itemize}
\item An action likely to lead to grave consequences like,
\item Data loss may lead to commercial loss.
\item Confidentiality breached.
\item Political issues\dots{}
\item Network breakdown lead to service and information flow disruption.
\end{itemize}
\rule[0.5ex]{1\columnwidth}{1pt}

\section{RESPONSE: }
\begin{itemize}
\item An act of responding.
\item Something constituting a reply or a reaction.
\item The activity or inhibition of previous activity or any of its parts
resulting from stimulation
\item The output of a transducer or detecting device resulting from a given
input.
\item Ideally Incident Response would be a set of policies that allow an
individual or individuals to react to an incident in an efficient
and professional manner thereby decreasing the likelihood of grave
consequences.
\end{itemize}
\rule[0.5ex]{1\columnwidth}{1pt}

\section{ISO 17799 }
\begin{itemize}
\item Outlines Comprehensive Incident Response and Internal Investigation
Procedures
\item Detailed Provisions on Computer Evidence Preservation and Handling 
\end{itemize}
\rule[0.5ex]{1\columnwidth}{1pt}

\section{PURPOSE OF INCIDENT RESPONSE:}
\begin{itemize}
\item Minimize overall impact.
\item Hide from public scrutiny.
\item Stop further progression.
\item Involve Key personnel.
\item Control situation.
\end{itemize}
%

\subsection{Minimize overall impact through,}
\begin{itemize}
\item Recover Quickly \& Efficiently.
\item Respond as if going to prosecute.
\item If possible replace system with new one.
\item Priority one, business back to normal.
\item Ensure all participants are notified.
\item Minimize overall impact.
\item Recover Quickly \& Efficiently.
\item Secure System.
\item Lock down all known avenues of attack.
\item Assess system for unseen vulnerabilities.
\item Implement proper auditing.
\item Implement new security measures. 
\item Follow-up (A continuous process)
\item Ensure that all systems are secure.
\item Continue prosecution.
\item Securely store all evidence and notes.
\item Distribute lessons learned.
\end{itemize}

\subsection{IDENTIFYING FALSE POSITIVES:}

Each event that detection tools generate falls into one of four categories,
depending on whether alert fired and whether something bad actually
happened:

\subsubsection{TRUE POSITIVES: Something bad happened, and the system caught it.}

\subsubsection{TRUE NEGATIVES: The activity is benign (gentle, kind), and no alert
has been generated.}

\subsubsection{FALSE POSITIVES: The system alerts, but the activity was not actually
malicious.}

\subsubsection{FALSE NEGATIVES: Something bad happened, but the system did not catch
it.}

Tools do not always alert when something bad happens, and just because
they throw an alarm does not necessarily mean it is time to isolate
a host or call the police. 

Example:

Just because an IDS alerted that \textquotedblleft the Web server
has been hacked\textquotedblright{} does not mean that the Web server
was actually hacked.

\rule[0.5ex]{1\columnwidth}{1pt}

\section{SIGNAL TO NOISE RATIO: }
\begin{quotation}
\noindent \textcolor{red}{EVERY ALERT IS IMPORTANT.}
\end{quotation}

\subsection{To illustrate, consider the case of SOC A and SOC B. }

In SOC A, the daily work queue contains approximately 100 reliable,
high-fidelity, usable alerts. Each one is reviewed by an analyst.
If incident response is necessary for a given alert, it is performed.

In SOC B, the daily work queue contains approximately 1,00,000 alerts,
almost all of which are false positives. Analysts attempt to review
them according to priority. 

Because of the large volume (even for alerts of the highest priority),
analysts cannot successfully review all of the highest-priority alerts. 

Additionally, because of the large number of false positives, SOC
B's analysts become desensitized to alerts and do not take them particularly
seriously.

\rule[0.5ex]{1\columnwidth}{1pt}

\part{TRIAGE:}

The triage process happens between the moment when an alert is triggered
and the time when an incident response process is initiated. 

Not every alert generated by a SIEM product triggers an incident,
some might prompt refinements of SIEM content or changes in security
policy. 

This process should include steps that allow security personnel to
unambiguously determine whether the alert is an indicator of an incident,
needs to be suppressed in the future, or requires further investigation
or escalation.

Triage is the assessment of a security event to determine if there
is a security incident its priority and the need for escalation.

As it relates to potential malware incidents the purpose of triaging
may vary. 

A few potential questions triaging may address are:
\begin{itemize}
\item Is malware present on the system?
\item How did it get there?
\item What was it trying to accomplish?
\end{itemize}
\rule[0.5ex]{1\columnwidth}{1pt}

\section{SECURITY INCIDENT:}

Understanding whether an event is an actual incident is not very simple
and evident every time. 

\paragraph{\textcolor{red}{EVENT CORRELATION. }}

\paragraph*{Here are a few tips for the verification:}
\begin{itemize}
\item Adjacent Data -- Check the information adjacent to the event. For
example, if an endpoint has a virus signature hit, look to see if
there\textquoteright s evidence the virus is running before calling
for further response metrics.
\item Intelligence Review -- Understand the context around the intelligence.
Just because an IP address was flagged as part of a botnet last week
does not mean it still is part of a botnet today.
\item Initial Priority -- Align with operational incident priorities and
classify incidents appropriately. Make sure the right level of effort
is applied to each incident.
\item Cross Analysis -- Look for and analyze potentially shared keys, such
as IP addresses or domain names, across multiple data sources for
better data acuity.
\end{itemize}

\paragraph{Once an event is verified, the event becomes an investigation or
an incident. All incidents must be investigated and tracked.}

\rule[0.5ex]{1\columnwidth}{1pt}

\section{CLASSIFICATION OF TYPES OF INCIDENTS:}

Understand what types of attacks are likely to be used against your
organization. 

List of different attach by NIST is:

\subsubsection{External/Removable Media: An attack executed from removable media
(e.g., flash drive, CD) or a peripheral device.}

\subsubsection{Email: An attack executed via an email message or attachment (e.g.
malware infection).}

\subsubsection{Attrition: An attack that employs brute force methods to compromise,
degrade, or destroy systems, networks, or services. }

\subsubsection{Improper Usage: Any incident resulting from violation of an organization\textquoteright s
acceptable usage policies by an authorized user, excluding the above
categories.}

\subsubsection{Web: An attack executed from a website or a web-based application
(e.g. drive-by download).}

\subsubsection{Loss or Theft of Equipment: The loss or theft of a computing device
or media used by the organization, such as a laptop or smartphone.}

\subsubsection{Other: An attack that does not fit into any of the other categories.}

\rule[0.5ex]{1\columnwidth}{1pt}

\section{ACTION ON SECURITY INCIDENT:}

\subsection{INCIDENT SCENE SNAPSHOT:}
\begin{itemize}
\item Record state of computer
\item Photos, 
\item State of computer, 
\item What is on the screen?
\item What is obviously running on the screen?
\item Xterm?
\item X-windows?
\item Should you port scan the affected computer?
\end{itemize}
\begin{description}
\item [{Pros:}] You can see all active and listening ports.
\item [{Cons:}] It affects the computer and some backdoor log how many
connections come into them and could tip off the bad guy.
\end{description}

\subsection{UNPLUG POWER FROM SYSTEM:}

\paragraph{This method may be the most damaging to effective analysis though
there are some benefits.}
\begin{description}
\item [{Pros:}] Benefits include that you can now move the system to a
more secure location and that you can physically remove the hard drive
from the system
\item [{Cons:}] You lose evidence of all running processes and memory.
\end{description}
%

\subsection{UNPLUG FROM THE NETWORK:}
\begin{itemize}
\item Unplug it from the network and plug the distant end into a small hub
that is not connected to anything else.
\item Most systems will write error messages into log files if not on a
network. 
\item If you make the computer think it is still on a network, you will
succeed in limiting the amount of changes to that system.
\end{itemize}

\subsection{BACKUP OR ANALYZE:}

\paragraph{Set up in policy for your incident response:}
\begin{itemize}
\item It depends on the system and what you need it for.
\item To get BEST evidence BACKUP first at the cost of time to get answers.
\item To get FAST answers ANALYZE first at the cost of getting best evidence.
\item Label systems with priority. Some will need answers quicker than your
ability to get best evidence.
\end{itemize}
\rule[0.5ex]{1\columnwidth}{1pt}


\section{PRIORITIZING INCIDENTS:}
\begin{itemize}
\item An Incident's priority is usually determined by assessing its impact
and urgency.
\item Urgency is a measure how quickly a resolution of the Incident is required.
\item Impact is measure of the extent of the Incident and of the potential
damage caused by the Incident before it can be resolved.
\end{itemize}
\rule[0.5ex]{1\columnwidth}{1pt}

\pagebreak{}

\section{INCIDENT PRIORITIZING TABLE:}

\begin{tabular}{|c|l|}
\hline 
\textbf{CATEGORY} & \textbf{DESCRIPTION}\tabularnewline
\hline 
\hline 
\multirow{4}{*}{\textbf{\textcolor{red}{\Large{}HIGH}}} & The damage caused by the incident increases rapidly. \tabularnewline
\cline{2-2} 
 & Work that cannot be completed by the staff is highly time sensitive.\tabularnewline
\cline{2-2} 
 & A minor incident can be prevented from becoming a major incident by
acting immediately.\tabularnewline
\cline{2-2} 
 & Several users with VIP status are affected.\tabularnewline
\hline 
\multirow{2}{*}{\textbf{\textcolor{orange}{\Large{}MEDIUM}}} & The damage caused by the incident increases considerably over time. \tabularnewline
\cline{2-2} 
 & A single user with VIP status is affected. \tabularnewline
\hline 
\multirow{2}{*}{\textbf{\textcolor{teal}{\Large{}LOW}}} & The damage caused by the incident only marginally increases over time. \tabularnewline
\cline{2-2} 
 & Work that cannot be completed by staff is not time sensitive.\tabularnewline
\hline 
\end{tabular}
\end{document}
