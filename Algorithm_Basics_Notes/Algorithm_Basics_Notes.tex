\documentclass[10pt,british]{article}
\usepackage[T1]{fontenc}
\usepackage[latin9]{luainputenc}
\usepackage[a4paper]{geometry}
\geometry{verbose,tmargin=2cm,bmargin=2cm,lmargin=2cm,rmargin=2cm}
\usepackage{babel}
\usepackage{setspace}
\usepackage{microtype}
\onehalfspacing
\usepackage[unicode=true,pdfusetitle,
 bookmarks=true,bookmarksnumbered=true,bookmarksopen=false,
 breaklinks=false,pdfborder={0 0 0},pdfborderstyle={},backref=section,colorlinks=false]
 {hyperref}
\makeatletter
\@ifundefined{date}{}{\date{}}
\usepackage{fontspec}
\AtBeginDocument{
  \def\labelitemi{\large\(\star\)}
}

\AtBeginDocument{
  \def\labelitemi{\(\star\)}
}

\makeatother

\begin{document}
\title{\textbf{\huge{}INTRODUCTION TO ALGORITHMS AND PROBLEM SOLVING}}
\author{AGNI DATTA}
\maketitle
\begin{center}
\rule[0.5ex]{0.5\columnwidth}{0.5pt}
\par\end{center}

\pagebreak{}

\tableofcontents{}

\vfill{}

\pagebreak{}

\part{Theory}

\section{Algorithm:}

\textbf{An algorithm is a sequence of unambiguous instructions for
solving a problem, i.e., for obtaining a required output for any legitimate
input in a finite amount of time. An Algorithm, therefore, corresponds
to a solution to a problem that is independent of any programming
language.}

A well-defined computational procedure that takes some value or a
set of values as input and procedures to find the output. It represents
the solution to a problem in a step wise manner.
\begin{verse}
\rule[0.5ex]{1\columnwidth}{1pt} 
\end{verse}

\section{Program:}

\textbf{A set of instructions is known as a program. A program is
an expression of an algorithm in a programming language.}
\begin{verse}
\rule[0.5ex]{1\columnwidth}{1pt} 
\end{verse}

\section{Characteristics Of An Algorithm:}
\begin{itemize}
\item Finiteness i.e., the number of steps.
\item Definiteness i.e., the steps should be well defined.
\item Generality i.e., the program should be generic and be able to solve
a certain class of problems.
\item Effectiveness i.e., the code should be basic enough to be easily understandable.
\item IO the code should be able to take inputs and produce outputs.
\end{itemize}
\begin{verse}
\rule[0.5ex]{1\columnwidth}{1pt} 
\end{verse}

\section{Procedure Of Writing An Algorithm:}
\begin{enumerate}
\item Breaking the problem into sub problems.
\item Choose a suitable data structure.
\item Constructions of loops if necessary.
\item Establishing the initial condition for loops.
\item Finding the iterative construct.
\item Termination of loops.
\end{enumerate}
This is also known as the top-down design.
\begin{verse}
\rule[0.5ex]{1\columnwidth}{1pt} 
\end{verse}

\section{General Problem-solving Methods:}
\begin{itemize}
\item Brute Force,
\item Divide and Conquer,
\item Decrease and Conquer,
\item Transform and Conquer,
\item Greedy Programming,
\item Backtracking,
\item Branch and Bound,
\item Approximation Algorithms,
\item Space and Time Trade-offs,
\item Dynamic Programming,
\item Exhaustive Search.
\end{itemize}
\begin{verse}
\rule[0.5ex]{1\columnwidth}{1pt} 
\end{verse}

\section{Pseudo Code:}

Pseudo Code is a plain language description of the steps in an algorithm
or another system. Pseudo Code often uses structural conventions of
a normal programming language, but is intended for human reading rather
than machine reading. The immediate representation of a problem between
an algorithm and a program. Pseudo Code is a method of representing
an algorithm. There are a lot of formats used for the representation
of pseudo code and most of them borrow some of the structures from
popular programming languages.

Some programming constructs used for pseudo code include:

\noindent %
\noindent\begin{minipage}[t]{1\columnwidth}%
\begin{itemize}
\item \texttt{read} 
\item \texttt{print} 
\item \texttt{set} 
\item \texttt{initialize} 
\item \texttt{if-then-end if} 
\item \texttt{if-then-else-end if} 
\item \texttt{repeat-until} 
\item \texttt{increment} 
\item \texttt{goto} 
\end{itemize}
%
\end{minipage}

\ 

Pseudo Code closely relates to a programming language.
\begin{verse}
\rule[0.5ex]{1\columnwidth}{1pt} 
\end{verse}

\section{Flowchart:}

A schematic representation of a sequence of operations, as in a manufacturing
process or computer program.

or

A graphical representation of the sequence of operations in an information
system or program.
\begin{itemize}
\item Information system flowcharts show how data flows from source documents
through the computer to final distribution to users.
\item Program flowcharts show the sequence of instructions in a single program
or subroutine.
\item Flow charts is a graph used to depict or show a step-by-step solution
using symbols which represent a task.
\item The symbols used consists of geometrical shapes that are connected
by flow lines. It is an alternative to pseudo coding; whereas a pseudo
code description is verbal, a flow chart is graphical.
\end{itemize}
\begin{verse}
\rule[0.5ex]{1\columnwidth}{1pt} 
\end{verse}

\section{Programming Languages:}

A programming language is a formal language comprising a set of strings
that produce various kinds of machine code output.

It is a tool for developing executable models for a class of problem
domains.

\subsection{Levels Of Programming Language:}

\subsubsection{Low-level Language:}

Machine Level Language this includes instructions in 1s and 0s i.e.,
in binary.
\begin{description}
\item [{Example:}] Binary Language.
\end{description}

\subsubsection{Middle-level Language:}

Assembly Level Language uses mnemonics to create instructions.
\begin{description}
\item [{Example:}] Assembly Level Language.
\end{description}

\subsubsection{High-level Language:}

Very similar to human-level language.
\begin{description}
\item [{Examples:}] COBOL, FORTRAN, BASIC, C, JAVA
\end{description}
\begin{verse}
\rule[0.5ex]{1\columnwidth}{1pt} 
\end{verse}

\section{Types Of Programming Language:}
\begin{itemize}
\item Batch Programming (BASIC, FORTRAN, COBOL)
\item Structured Programming (PASCAL, C)
\item Object-Oriented Programming (C++, Java, C\#, PYTHON)
\item Logic/Declarative
\item Programming (Prolog)
\item Functional/Applicative Programming (Lisp)
\end{itemize}
\begin{verse}
\rule[0.5ex]{1\columnwidth}{1pt} 
\end{verse}

\section{High-level Languages:}

\subsection{C:}
\begin{itemize}
\item Developed by Bell Laboratories in the early 1970s. 
\item Provides control and efficiency of assembly language while having
third-generation language features. 
\item Often used for system programs. 
\item UNIX is written in C. 
\end{itemize}

\subsection{C++:}
\begin{itemize}
\item It is C language with additional features. 
\item Widely used for developing system and application software. 
\item Graphical user interfaces can be developed easily with visual programming
tools. 
\end{itemize}

\subsection{HTML:}
\begin{itemize}
\item Hypertext Markup Language. 
\item Used on the Internet and the World Wide Web 
\item Web page developer puts brief codes called tags in the page to indicate
how the page should be formatted. 
\end{itemize}

\subsection{XML:}
\begin{itemize}
\item Extensible Markup Language. 
\item A language for defining other languages 
\end{itemize}

\subsection{JAVA:}
\begin{itemize}
\item An object-oriented language similar to C++ that eliminates lots of
C++\textquoteright s problematic features.
\item Allows a web page developer to create programs for applications.
\item The objective of JAVA developers is that it be a machine, platform
and operating system independent.
\end{itemize}
\begin{verse}
\rule[0.5ex]{1\columnwidth}{1pt} 
\end{verse}

\section{Difference Between A High And Low-level Programming Language:}
\begin{itemize}
\item Low-level languages work more closely with hardware and do not require
a compiler to be executed.
\item High-level languages are more understandable for the programmer in
terms of the words in the code.
\end{itemize}
\begin{verse}
\rule[0.5ex]{1\columnwidth}{1pt} 
\end{verse}

\section{Verification And Validation:}

The program being developed must be checked to ensure that it meets
its specification and delivers the functionality expected by the people
paying for the software.

\subsection{Verification:}

Are you building the product, right? The software must conform to
its specification.

\subsection{Validation:}

Are you building the right product? The software should do what the
user requires.
\begin{verse}
\rule[0.5ex]{1\columnwidth}{1pt} 
\end{verse}

\section{Efficiency Of Algorithms:}

Every algorithm must use up some of the computer resources to complete
its tasks. The resources most relevant to efficiency are central processor
time and internal memory. The efficiency or running time of an algorithm
is stated as a function relating the input length to the number of
steps, known as time complexity, or volume of memory, known as space
complexity. A good algorithm is correct, but a great algorithm is
both correct and efficient. The most efficient algorithm is one that
takes the least amount of execution time and memory usage possible
while still yielding a correct answer. Algorithm efficiency is used
to describe properties of an algorithm relating to how much of various
types of resources it consumes. The run time complexity of an algorithm
is the amount of time that it takes to complete once it has begun.
The space complexity of an algorithm is the amount of storage space
that it requires while running from start to completion. 
\begin{verse}
\rule[0.5ex]{1\columnwidth}{1pt} 
\end{verse}

\section{Parameter:}

A parameter is a piece of data provided as input to a function or
procedure. 
\begin{verse}
\rule[0.5ex]{1\columnwidth}{1pt} 
\end{verse}

\section{Analysis Of Algorithms:}

Quantitative measures are valuable in that they can give us a way
of directly predicting the performance of an algorithm and of comparing
the relative performance of two or more algorithms that are intended
to solve the same problem. This can be important because the use of
an algorithm that is more efficient means saving in computing resources
which translates into a saving in time and memory.

We can have three cases to analyze an algorithm:

\subsection{Worst Case:}

$n$ the worst case analysis, we calculate upper bound on running
time of an algorithm. We must know the case that causes maximum number
of operations to be executed. For Linear Search, the worst case happens
when the element to be searched ($x$ in the above code) is not present
in the array. When $x$ is not present, the search$(\:)$ functions
compares it with all the elements of $arr[\:\:]$ one by one. Therefore,
the worst case time complexity of linear search would be $\Theta(n)$.

\subsection{Average Case:}

In average case analysis, we take all possible inputs and calculate
computing time for all of the inputs. Sum all the calculated values
and divide the sum by total number of inputs. We must know (or predict)
distribution of cases. For the linear search problem, let us assume
that all cases are uniformly distributed (including the case of $x$
not being present in array). So we sum all the cases and divide the
sum by $(n+1)$. Following is the value of average case time complexity

\subsection{Best Case:}

In the best case analysis, we calculate lower bound on running time
of an algorithm. We must know the case that causes minimum number
of operations to be executed. In the linear search problem, the best
case occurs when $x$ is present at the first location. The number
of operations in the best case is constant (not dependent on $n$).
So time complexity in the best case would be $\Theta(1)$ Most of
the times, we do worst case analysis to analyze algorithms. In the
worst analysis, we guarantee an upper bound on the running time of
an algorithm which is good information. The average case analysis
is not easy to do in most of the practical cases and it is rarely
done. In the average case analysis, we must know (or predict) the
mathematical distribution of all possible inputs. The Best Case analysis
is useless. Guaranteeing a lower bound on an algorithm does not provide
any information as in the worst case, an algorithm may take years
to run. For some algorithms, all the cases are asymptotically same,
i.e., there are no worst and best cases. 
\begin{verse}
\rule[0.5ex]{1\columnwidth}{1pt} 
\end{verse}

\end{document}
